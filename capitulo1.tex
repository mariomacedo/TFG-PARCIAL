\chapter[Introdução]{Introdução}
\label{cap:cap1}
\par
De acordo com a \acrfull{onu} (2018), promover a participação cidadã é fundamental para a governança de uma sociedade inclusiva.
O objetivo dessa participação deve ser composto pela melhoraria do acesso à informação e serviços públicos,
e pelo incentivo a inclusão do cidadão nas tomadas de decisões públicas que impactem o bem estar da sociedade como um todo, e do indivíduo em particular. 
As funções de fazer política e entregar serviços públicos precisam ser reinterpretadas e os cidadãos têm que se envolver nos processos políticos \cite{bovaird2007beyond}. 
O uso de \acrfull{tic} no setor público tem transformado a governança global, nesse sentido, exigindo que governos forneçam
serviços melhores e mais eficientes a organizações e indivíduos \cite{afdb2014uneca}. O conceito de participação eletrônica foi definido por \cite{macintosh2008democracy} como:
"O uso de informações e comunicação tecnológicas a fim de ampliar e aprofundar a participação política para que cidadãos sejam capazes de se conectar 
uns aos outros e aos seus representantes eleitos".
A noção de governança digital é relatada por \cite{reddick2012public} como o resultado da adoção de infraestruturas tecnológicas para a otimização da entrega de serviços à população.
Ou seja, a utilização de \acrshort{tic} pelo setor público, passa a se referir ao modo como a internet pode melhorar a capacidade do Estado de formular
e implementar suas políticas \cite{parra2017governancca}. Sendo assim, \cite{germani2016desafios} complementa a definição de governança digital dizendo que:

\hspace{4cm}
\begin{minipage}{.66\textwidth}		
    \begin{singlespace}
        \fontsize{10}{12}\selectfont A utilização pelo setor público de recursos de tecnologia da informação e comunicação com o objetivo de melhorar a disponibilização
        de informação e a prestação de serviços públicos,
        incentivar a participação da sociedade no processo de tomada de decisão e aprimorar os níveis de responsabilidade, transparência e efetividade do governo.    
        \end{singlespace}
\end{minipage}

\vspace{0.3CM}
\par
Para \cite{braga2016participaccao} a possibilidade de um maior acesso as informações e ao conhecimento proporcionado pelas \acrshort{tic}
permite uma maior transparência nas decisões tomadas por governantes. A alegação que \cite{vaz2017transformaccoes} apresenta é a existência da necessidade de alteração do atual modelo
\textit{broadcasting} de governança eletrônica e quebra do monopólio estatal sobre as decisões e iniciativas de transparência e participação popular nas políticas públicas.
Para que haja a evolução desse modelo, \cite{o2011government} sugere que o governo deve disponibilizar suas informações em uma infraestrutura que permita sua sistemática reutilização
pela sociedade. Ao disponibilizar suas informação, o governo estimula o desenvolvimento de ferramentas tecnológicas e amplia a possibilidade 
do uso diverso das informações \cite{zuiderwijk2012socio}.

%==================================================================================================================================================================================================
%Motivação (quantidade de recursos relacionados à participação eletrônica)
\section{Motivação}
\label{sec:motivacao}
Segundo \cite{vaz2017transformaccoes}, o crescimento da comunidade \textit{open source} atrelado ao compartilhamento de dados governamentais abertos, permite a produção
descentralizada de aplicações, serviços e sistemas tecnológicos. Com isso, cria-se o que o mesmo autor chama de "segunda geração de governança eletrônica", que consiste na busca
por iniciativas além das unidirecionais, onde o governo apenas disponibiliza os dados captados sobre a população.

\par
O conceito de governança \textit{open source} democratiza a tomada de decisão e incentiva a colaboração voluntária entre os indivíduos \cite{rushkoff2003open}.
Diversos paradigmas, sobre o modelo de tomada de decisão, vêm sendo reexaminados, o papel do gestor público e do cidadão têm sido reavaliados,
e o uso das ferramentas computacionais se disseminado em muitos ambientes diferentes \cite{medeiros2009novos}.

\par
Devido aos incentivos gerados por esses novos paradigmas, muitas ferramentas vêm sendo desenvolvidas focadas nesse escopo.
Essas ferramentas podem ter aplicações direcionadas a cidades, bairros, regiões, estados e/ou países.  
Alguns exemplos podem ser encontrado na Tabela \ref{tab:1-ferramentas}.

\addtocounter{table}{-1}
\begin{table}[!ht]
    \centering
    \caption{Ferramentas de Participação Eletrônica}
    \label{tab:1-ferramentas}
    \begin{tabular}{l*{2}{>{\raggedright\arraybackslash}p{0.5\linewidth}}}
    \toprule
        Nome                             & Acesso                           \\ 
    \midrule
    % spellcheck-language "en"
        Crowd For Roads                  & c4rs.eu                          \\
        Decidim Barcelona                & decidim.barcelona                \\
        e-Cidadania                      & senado.leg.br/ecidadania         \\
        Iniciativa de Cidadania Europeia & ec.europa.eu/citizens-initiative \\
        LabRIO                           & lab.rio                          \\
        Mandato Participativo            & saopaulo.sp.leg.br               \\
        Participa.br                     & participa.br                     \\
        Patio                            & patiolla.fi                      \\
        Plataforma Brasil                & plataformabrasil.org.br          \\
        Portal e-Democracia              & edemocracia.camara.gov.br        \\
        SigaLei                          & sigalei.com.br                   \\
        Vote na Web                      & votenaweb.com.br                 \\ 
        We The People                    & petitions.whitehouse.gov         \\
    \bottomrule
    \end{tabular}
\end{table}

As novas tecnologias mostraram uma gama de possibilidades para que os cidadãos ampliem o peso de sua participação nas decisões políticas,
melhorando a capacidade de mobilização, articulação, e possibilitando um maior envolvimento dos atores sociais \cite{araujo2015democracia}.

\par
\cite{saebo2008shape} divide os atores em quatro grupos:\\

\begin{minipage}{.66\textwidth}	
    I) Cidadãos \\
    II) Governantes \\
    III) Instituições estatais \\
    IV) Instituições Voluntárias. \\
\end{minipage}


%==================================================================================================================================================================================================
%Justificativa (uma classificação pode apoiar o entendimento nessa área)
\section{Justificativa}
\label{sec:justificativa}

%==================================================================================================================================================================================================
%Objetivo do seu trabalho (ferramenta que apoia a edição colaborativa da taxonomia)
\section{Objetivo}
\label{sec:objetivo}

%==================================================================================================================================================================================================
%Contribuições (prático (gestor e cidadão comum) e pesquisador)
\section{Contribuições}
\label{sec:contribuicoes}

%==================================================================================================================================================================================================
%Estrutura da monografia
\section{Organização do trabalho}
\label{sec:organizacao}