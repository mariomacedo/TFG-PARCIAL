\chapter[Introdução]{Introdução}
\label{cap:cap1}
\par
Uma tendência que vem sendo observada é a tentativa de aproximação entre os cidadãos e seus governantes, um exemplo disso são as iniciativas de governos abertos.
A agenda de 2030 da \acrfull{onu} para a transformação do mundo através do desenvolvimento sustentável, deixa claro, em seu parágrafo 48, que
indicadores estão sendo desenvolvidos para ajudar nesse trabalho de aproximar governo e cidadão. Dados desagregados de qualidade, acessíveis,
atualizados e confiáveis serão necessários para colaborar com a análise do progresso das iniciativas sobre participação. 
Esses dados são de suma importância para a tomada de decisões \cite{assembly2015transforming}.

\par
A \acrshort{onu} define o conceito de participação eletrônica como "o processo de engajar cidadãos através de \acrfull{tic} em política, tomadas de decisão e
serviços públicos, fazendo com que este processo seja inclusivo, participativo e deliberado". Sendo assim, \citeonline{braga2016participaccao} mostra
que o uso de ferramentas de participação eletrônica torna-se fundamental para o cumprimento das metas traçadas em 2015. 

%==================================================================================================================================================================================================
%Motivação (quantidade de recursos relacionados à participação eletrônica)
\par
A \acrfull{ocde}, juntamente com a \acrshort{onu} e a \acrfull{ue} criaram \textit{frameworks} para a avaliação do \textit{status} do governo eletrônico,
através de indicadores \cite{onu2018}. A metodologia utilizada permite medir a eficácia do governo eletrônico na prestação de serviços públicos 
e identificar padrões de desenvolvimento e desempenho.
Atualmente, governantes podem encontrar diretrizes para seguir na hora de desenvolver uma ferramenta de participação eletrônica. 
Por exemplo o trabalho de \citeonline{scherer2010hands}, que demonstra a metodologia utilizada para a criação de duas ferramentas de participação eletrônica, a VoicE e a VoiceS, na Europa.

%==================================================================================================================================================================================================
%Justificativa (uma classificação pode apoiar o entendimento nessa área)
Por incentivo do projeto \acrfull{e-uropa}, \citeonline{kinyik2015guideline} apresenta um modelo de classificação de ferramentas de participação eletrônica,
representado na Tabela \ref{tab:classificacao}. 
\addtocounter{table}{-1}
\begin{table}[!ht]
    \centering
    \caption{Classificação das petições da \acrshort{ice}}
    \label{tab:classificacao}
    \begin{tabular}{l*{2}{>{\raggedright\arraybackslash}p{0.5\linewidth}}}
    \toprule
        Nome                             \\
    \midrule
        Abordagem                        \\
        \midrule
        Nível de Engajamento             \\
        Estágios das políticas           \\
        Tempo de duração                 \\
        Atores                           \\
        Tecnologia Usadas                \\
        Idiomas                          \\
        Meio de Acesso                   \\
        Regras de engajamento            \\
        Visor Urbano                     \\
        \textit{Feedback}                \\
    \bottomrule
    \end{tabular}
\end{table}
\newpage

\acrshort{e-uropa} tem o objetivo de aumentar e disseminar o conhecimento, nos cidadãos europeus, sobre ferramentas de participação eletrônica, alegando que 
fazendo assim, as demandas da sociedade seriam melhores e mais rapidamente atendidas.

%==================================================================================================================================================================================================
%Objetivo do seu trabalho (ferramenta que apoia a edição colaborativa da taxonomia)
Sendo assim, este projeto tem o objetivo de modelar, desenvolver e implementar uma aplicação \textit{web} para apoiar a edição colaborativa de uma taxonomia sobre ferramentas
de participação eletrônica. A aplicação permitirá a interação dos usuários de maneira intuitiva e amigável, colaborando para a edição das informações apresentadas.
Essa aplicação apresentará tanto a taxonomia, quanto as ferramentas de participação eletrônica classificadas pelos usuários.

%==================================================================================================================================================================================================
%Contribuições (prático (gestor e cidadão comum) e pesquisador)
Espera-se construir uma ferramenta que contribua tanto no âmbito teórico quanto no âmbito prático do assunto abordado. 
\par
No que diz respeito ao campo teórico, é desejado que a ferramenta possa ajudar os pesquisadores a entender mais sobre ferramentas de participação eletrônica e suas classificações. 
Espera-se que com a utilização de critérios bem estabelecidos pela literatura atual, o entendimento da classificação taxonômica gerada seja facilitado.
\par
Por outro lado, tanto para o gestor público, quanto para o cidadão comum, o uso da aplicação pode facilitar a escolha de uma ou mais ferramentas de participação
eletrônica. Essa escolha poderá levar em conta critérios isolados ou um aglomerado de critérios, dependendo da demanda do usuário.
