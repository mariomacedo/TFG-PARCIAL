\chapter[Introdução]{Introdução}
\label{cap:cap1}
\par
Uma tendência que vem sendo observada é a tentativa de aproximação entre os cidadãos e seus governantes, um exemplo disso são as iniciativas de governos abertos.
A agenda de 2030 da \acrfull{onu} para a transformação do mundo através do desenvolvimento sustentável, deixa claro em seu parágrafo 48 que:

\hspace{4cm}
\begin{minipage}{.66\textwidth}		
    \begin{singlespace}
        \fontsize{10}{12}\selectfont Indicadores estão sendo desenvolvidos para ajudar neste trabalho. Dados desagregados de qualidade, acessíveis, 
        atualizados e confiáveis serão necessários para ajudar na medição do progresso e para garantir que ninguém seja deixado para trás. 
        Esses dados são a chave para a tomada de decisões. Dados e informações disponíveis em mecanismos de comunicação devem ser usados sempre que possível.
        \cite{assembly2015transforming}
        \end{singlespace}
\end{minipage}
\vspace{0.3CM}

A \acrshort{onu} também define o conceito de participação eletrônica como "o processo de engajar cidadãos através de \acrfull{tic} em política, tomadas de decisão e
serviços públicos, fazendo com isso que este processo seja inclusivo, participativo e deliberado". Sendo assim, o uso de ferramentas de participação eletrônica torna-se
fundamental para o cumprimento das metas traçadas em 2015. 

%==================================================================================================================================================================================================
%Motivação (quantidade de recursos relacionados à participação eletrônica)
\section{Motivação}
\label{sec:motivacao}
\par
A \acrfull{ocde}, juntamente com a \acrshort{onu} e a \acrfull{ue} criaram \textit{frameworks} para a avaliação das ferramentas através de indicadores. 
Atualmente, governantes podem encontrar diretrizes para seguir na hora de desenvolver uma atividade de participação eletrônica.

\par
Contudo, não foi encontrada uma taxonomia focada em classificar as ferramentas de participação, justificando assim, a elaboração deste trabalho.

%==================================================================================================================================================================================================
%Justificativa (uma classificação pode apoiar o entendimento nessa área)
\section{Justificativa}
\label{sec:justificativa}
Através da \acrfull{ice}, \citeonline{kinyik2015guideline} apresenta um modelo de classificação de ferramentas de participação eletrônica
representado na Tabela \ref{tab:classificacao}. 
\addtocounter{table}{-1}
\begin{table}[!ht]
    \centering
    \caption{Classificação das petições da \acrshort{ice}}
    \label{tab:classificacao}
    \begin{tabular}{l*{2}{>{\raggedright\arraybackslash}p{0.5\linewidth}}}
    \toprule
        Nome                             \\ 
    \midrule
        Abordagem                        \\
        Nível de Engajamento             \\
        Estágios das políticas           \\
        Tempo de duração                 \\
        Atores                           \\
        Tecnologia Usadas                \\
        Idiomas                          \\
        Meio de Acesso                   \\
        Regras de engajamento            \\
        Visor Urbano                     \\
        \textit{Feedback}                \\
    \bottomrule
    \end{tabular}
\end{table}
\newpage

%==================================================================================================================================================================================================
%Objetivo do seu trabalho (ferramenta que apoia a edição colaborativa da taxonomia)
\section{Objetivo}
\label{sec:objetivo}
Este projeto tem o objetivo de modelar, desenvolver e implementar uma aplicação \textit{web} para apoiar a edição colaborativa de uma taxonomia sobre ferramentas
de participação eletrônica. A aplicação permitirá a interação dos usuários de maneira intuitiva e amigável, colaborando para a edição das informações apresentadas.

%==================================================================================================================================================================================================
%Contribuições (prático (gestor e cidadão comum) e pesquisador)
\section{Contribuições}
\label{sec:contribuicoes}
Espera-se construir uma ferramenta que contribua tanto no âmbito teórico quanto no âmbito prático do assunto abordado. 
\par
No que diz respeito ao campo teórico, é desejado que a ferramenta possa ajudar os pesquisadores a entender mais sobre ferramentas de participação eletrônica e suas classificações. 
Espera-se que com a utilização de critérios bem estabelecidos pela literatura atual, o entendimento da classificação taxonômica gerada seja facilitado.
\par
Por outro lado, tanto para o gestor público, quanto para o cidadão comum, o uso da aplicação pode facilitar a escolha de uma ou mais ferramentas de participação
eletrônica. Essa escolha poderá levar em conta critérios isolados ou um aglomerado de critérios, dependendo da demanda do usuário.

%==================================================================================================================================================================================================
%Estrutura da monografia
\section{Organização do trabalho}
\label{sec:organizacao}
No desenvolvimento deste trabalho, uma aplicação será implementada para representar a taxonomia desenvolvida no contexto do projeto VisPublica.
Um experimento de validação será realizado para a análise sobre a utilização da taxonomia na tomada de decisão sobre as ferramentas de participação eletrônica.
Na Tabela \ref{tab:cronograma} estão as atividades previstas para a conclusão do trabalho, 
\begin{enumerate}
    \begin{singlespace}
    \fontsize{10}{12}\selectfont 
    \item\label{revisao}Revisão Bibliográfica
    \item\label{requisitos} Levantamento de requisitos funcionais do sistema
    \item\label{prototipo} Protótipo não funcional do sistema
    \item\label{modelagem} Modelagem do sistema
    \item\label{dev} Desenvolvimento e codificação
    \item\label{teste} Testes iniciais
    \item\label{validacao} Validação do sistema
    \item\label{expo} Exposição dos Resultados
    \end{singlespace}
\end{enumerate}

\definecolor{midgray}{RGB}{90,90,90}
\begin{table}[!ht]
\caption{Cronograma de Atividades}
\label{tab:cronograma}
\centering
    \begin{tabular}{|c|c|c|c|c|c|}
    \hline
    &\multicolumn{5}{c|}{2018}\\
    \hline
    &AGO&SET&OUT&NOV&DEZ\\
    \hline
    \ref{revisao}&\cellcolor{midgray}&&&&\\
    \hline
    \ref{requisitos}&\cellcolor{midgray}&\cellcolor{midgray}&&&\\
    \hline	
    \ref{prototipo}&&\cellcolor{midgray}&&&\\
    \hline			
    \ref{modelagem}&&\cellcolor{midgray}&\cellcolor{midgray}&&\\
    \hline	
    \ref{dev}&&\cellcolor{midgray}&\cellcolor{midgray}&\cellcolor{midgray}&\\
    \hline
    \ref{teste}&&&&\cellcolor{midgray}&\\
    \hline	
    \ref{validacao}&&&&\cellcolor{midgray}&\\
    \hline	
    \ref{expo}&&&&\cellcolor{midgray}&\cellcolor{midgray}\\
    \hline
    \end{tabular}
\end{table}
\vfill