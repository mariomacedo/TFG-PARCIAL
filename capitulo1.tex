\chapter[Introdução]{Introdução}
\par
De acordo com a \acrlong{onu} \cite{onu2018}, promover a participação cidadã é fundamental para a governança de uma sociedade inclusiva.
O objetivo dessa participação deve ser a melhoraria do acesso à informação e serviços públicos,
incentivar a inclusão do cidadão nas tomadas de decisões públicas que impactem o bem estar da sociedade como um todo, e do indivíduo em particular. 
As funções de fazer política e entregar serviços públicos precisam ser reinterpretadas e os cidadãos têm que se envolver nos processos políticos \cite{bovaird2007beyond}. 
O uso de \acrfull{tic} no setor público tem transformado a governança global, nesse sentido, exigindo que governos forneçam
serviços melhores e mais eficientes a organizações e indivíduos \cite{afdb2014uneca}. O conceito de participação eletrônica foi definido por \cite{macintosh2008democracy} como:
"O uso de informações e comunicação tecnológicas a fim de ampliar e aprofundar a participação política para que cidadãos sejam capazes de se conectar 
uns aos outros e aos seus representantes eleitos."
A noção de governança digital é relatada por \cite{reddick2012public} como o resultado da adoção de infraestruturas tecnológicas para a otimização da entrega de serviços à população.
Ou seja, a utilização de \acrshort{tic} pelo setor público, passa a se referir ao modo como a internet pode melhorar a capacidade do Estado de formular
e implementar suas políticas \cite{parra2017governancca}. Sendo assim, \cite{germani2016desafios} complementa a definição de governança digital dizendo que:

\hspace{4cm}
\begin{minipage}{.66\textwidth}		
    \begin{singlespace}
        \fontsize{10}{12}\selectfont A utilização pelo setor público de recursos de tecnologia da informação e comunicação com o objetivo de melhorar a disponibilização de informação e a prestação de serviços públicos,
        incentivar a participação da sociedade no processo de tomada de decisão e aprimorar os níveis de responsabilidade, transparência e efetividade do governo.    
        \end{singlespace}
\end{minipage}


    
\section{Motivação}

\section{Objetivo}

\section{Contribuições}

\section{Organização do trabalho}