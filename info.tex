% Informações de dados para CAPA e FOLHA DE ROSTO

\titulo{Uma ferramenta colaborativa para evolução de uma taxonomia}
\autor{Mário Guilherme Macedo}
\local{Itajubá}
\data{\today}
\orientador{Prof.\textsuperscript{a} Dr.\textsuperscript{a} Melise Maria Veiga de Paula}
\coorientador{Prof. Dr. Coorientador}
\instituicao{Universidade Federal de Itajubá - UNIFEI}
\def \siglaInstituicao{UNIFEI}
\def \orientadorTexto{Monografia realizada sob orientação da \imprimirorientador}
\def \programaLinhaUm{}
\def \programaLinhaDois{}
\def \programa{\programaLinhaUm \space \programaLinhaDois}
\def \areaconcentracao{Área de Concentração: Microeletrônica}

%\tipotrabalho{Tese (Doutorado)}
\tipotrabalho{Trabalho Final de Graduação}

% O preambulo deve conter o tipo do trabalho, o objetivo, o nome da instituição e a área de concentração 
\preambulo{Monografia apresentada como trabalho final de graduação, requisito parcial para obtenção do título de Bacharel em Ciência da Computação, sob orientação da \imprimirorientador.}

% aprovação
\def \diadeaprovacao{23}
\def \mesdeaprovacao{Outubro}
\def \anodeaprovacao{2018}

% frase utilizada na folha de aprovação
\def  \aprovacao{\large Dissertação aprovada por banca examinadora em \diadeaprovacao \space de \mesdeaprovacao \space de \anodeaprovacao, conferindo ao autor o título de \textbf {Mestre em Ciências em Engenharia Elétrica.}}

% Banca examinadora - Professores convidados
\def \professorConvidadoUm{Prof.(\textsuperscript{a}) Dr.(\textsuperscript{a}) Convidado(a)}
\def \professorConvidadoDois{Prof. Dr. Convidado2}
\def \professorConvidadoTres{Prof. Dr. Convidado3}
\def \professorConvidadoQuatro{Prof. Dr. Convidado4}
\def \professorConvidadoCinco{Prof. Dr. Convidado5}

% Banca examinadora
\def \bancaexaminadora{\professorConvidadoUm \\ \professorConvidadoDois \\ \professorConvidadoTres \\ \professorConvidadoQuatro \\ \professorConvidadoCinco}