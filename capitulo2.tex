\chapter[Revisão Bibliográfica]{Revisão Bibliográfica} 
\label{cap:cap2}
% Introdução do capítulo
Neste capítulo são apresentados os tópicos teóricos necessários para uma boa compreensão do desenvolvimento deste trabalho.

% 1a seção do capítulo 1
\section{Participação Eletrônica}
\label{sec:e-part}
A noção de governança digital é relatada por \citeonline{reddick2012public} como o resultado da adoção de infraestruturas tecnológicas para a otimização da entrega de serviços à população.
Ou seja, a utilização de \acrshort{tic} pelo setor público passa a se referir ao modo como a internet pode melhorar a capacidade do Estado de formular
e implementar suas políticas \cite{parra2017governancca}. Sendo assim, \citeonline{germani2016desafios} complementa a definição de governança digital afirmando que:

\hspace{4cm}
\begin{minipage}{.66\textwidth}		
    \begin{singlespace}
        \fontsize{10}{12}\selectfont É a utilização pelo setor público de recursos de tecnologia da informação e comunicação com o objetivo de melhorar a disponibilização
        de informação e a prestação de serviços públicos,
        incentivar a participação da sociedade no processo de tomada de decisão e aprimorar os níveis de responsabilidade, transparência e efetividade do governo.    
        \end{singlespace}
\end{minipage}

\vspace{0.3CM}
\par
Para \citeonline{braga2016participaccao}, a possibilidade de um maior acesso às informações e ao conhecimento proporcionado pelas \acrshort{tic}
permite uma maior transparência nas decisões tomadas por governantes. \citeonline{vaz2017transformaccoes} argumenta que há a necessidade de alteração do atual modelo
\textit{broadcasting} de governança eletrônica e quebra do monopólio estatal sobre as decisões e iniciativas de transparência e participação popular nas políticas públicas.
Para que haja a evolução desse modelo, \citeonline{o2011government} sugere que o governo deve disponibilizar suas informações em uma infraestrutura que permita sua sistemática reutilização
pela sociedade.Ao disponibilizar suas informação, o governo estimula o desenvolvimento de ferramentas tecnológicas e amplia a possibilidade 
do uso diverso dessas informações \cite{zuiderwijk2012socio}.
\par
Portanto, o uso de \acrfull{tic} no setor público tem transformado a governança global, exigindo que governos forneçam
serviços melhores e mais eficientes às organizações e indivíduos e ampliado as possibilidades de participação \cite{afdb2014uneca}.
De acordo com a \acrshort{onu}, promover a participação cidadã é fundamental para a governança de uma sociedade inclusiva.
O objetivo dessa participação deve ser definido pela melhoria do acesso à informação e aos serviços públicos
e pelo incentivo à inclusão do cidadão na tomada de decisão pública que impactam o bem estar da sociedade e do indivíduo. 
As funções de fazer política e entregar serviços públicos precisam ser reinterpretadas e os cidadãos têm que se envolver nos processos políticos \cite{bovaird2007beyond}.
\par
Essa junção entre a participação cidadã e a tecnologia resultou em uma nova modalidade de relacionamento entre governo e sociedade, a participação eletrônica. O conceito de participação eletrônica foi definido por \citeonline{macintosh2008democracy} como:
"O uso de informações e comunicação tecnológicas a fim de ampliar e aprofundar a participação política para que cidadãos sejam capazes de se conectar 
uns aos outros e aos seus representantes eleitos".
\par

\section{Ferramentas de Participação Eletrônica}
\label{sec:e-part tools}
Segundo \citeonline{vaz2017transformaccoes}, o crescimento da comunidade \textit{open source} atrelado ao compartilhamento de dados governamentais abertos permite a produção
descentralizada de aplicações, serviços e sistemas tecnológicos. Com isso, cria-se o que o mesmo autor chama de "segunda geração de governança eletrônica", que consiste na busca
por iniciativas além das unidirecionais, onde o governo apenas disponibiliza os dados captados sobre a população.

\par
O conceito de governança \textit{open source} democratiza a tomada de decisão e incentiva a colaboração voluntária entre os indivíduos \cite{rushkoff2003open}.
Diversos paradigmas, sobre o modelo de tomada de decisão, vêm sendo reexaminados, o papel do gestor público e do cidadão têm sido reavaliados,
e o uso das ferramentas computacionais se disseminado em muitos ambientes diferentes \cite{medeiros2009novos}.

\par
Devido aos incentivos gerados por esses novos paradigmas, muitas ferramentas vêm sendo desenvolvidas focadas nesse escopo.
Essas ferramentas podem ter aplicações direcionadas a cidades, bairros, regiões, estados e/ou países.  
Alguns exemplos podem ser encontrado na Tabela \ref{tab:ferramentas}.

\begin{table}[!ht]
    \centering
    \caption{Ferramentas de Participação Eletrônica}
    \label{tab:ferramentas}
    \begin{tabular}{l*{2}{>{\raggedright\arraybackslash}p{0.5\linewidth}}}
    \toprule
        Nome                             & Acesso                           \\ 
    \midrule
        Crowd For Roads                  & c4rs.eu                          \\
        Decidim Barcelona                & decidim.barcelona                \\
        e-Cidadania                      & senado.leg.br/ecidadania         \\
        Iniciativa de Cidadania Europeia & ec.europa.eu/citizens-initiative \\
        LabRIO                           & lab.rio                          \\
        Mandato Participativo            & saopaulo.sp.leg.br               \\
        Participa.BR                     & participa.br                     \\
        Patio                            & patiolla.fi                      \\
        Plataforma Brasil                & plataformabrasil.org.br          \\
        Portal e-Democracia              & edemocracia.camara.gov.br        \\
        SigaLei                          & sigalei.com.br                   \\
        Visor Urbano                     & visorurbano.com                  \\
        Vote na Web                      & votenaweb.com.br                 \\ 
        We The People                    & petitions.whitehouse.gov         \\
        WeLive                           & welive.eu                        \\
    \bottomrule
    \end{tabular}
\end{table}

\newpage

As novas tecnologias mostraram uma gama de possibilidades para que os cidadãos ampliem o peso de sua participação nas decisões políticas,
melhorando a capacidade de mobilização, articulação, e possibilitando um maior envolvimento dos atores sociais \cite{araujo2015democracia}.\\

\par
\citeonline{saebo2008shape} dividem os atores sociais em quatro grupos: 

\begin{minipage}{.66\textwidth}	
   \textit{I}) Cidadãos, \\
   \textit{II}) Governantes, \\
   \textit{III}) Instituições estatais e \\
   \textit{IV}) Instituições Voluntárias. \\
\end{minipage}

\par
Esse agrupamento feito pelo autor supracitado é uma abstração da análise de \citeonline{macintosh2006evaluating}, deixando de lado um quinto grupo:

\par
\textit{V}) Provedores de Tecnologia.

\par
Contudo, \citeonline{wimmer2007ontology} agrupa esses atores mais objetivamente, dividindo-os em apenas dois grupos, são eles:\\

\begin{minipage}{.75\textwidth}	
   \textit{a}) Beneficitários da utilização das ferramentas e \\
   \textit{b}) Responsáveis pela administração da ferramenta de participação.  \\
\end{minipage}

Esses grupos começaram a se utilizar de ferramentas de participação eletrônica, muitas vezes em projetos pilotos,
sem estratégias para medir os impactos gerados, as análises de performance ou as respostas dos cidadãos beneficiados \cite{macintosh2008democracy}.
Muitos autores chamam atenção para fato das ferramentas de participação serem predecessoras dos modelos de escolha e classificação para elas 
\cite{millard2006egovernance,macintosh2008democracy,reddick2012public}.

\par
A \acrfull{ocde}, juntamente com a \acrshort{onu} e a \acrfull{ue} criaram \textit{frameworks} para a avaliação das ferramentas através de indicadores. 
Atualmente, governantes podem encontrar diretrizes para seguir na hora de desenvolver uma atividade de participação eletrônica.

%==================================================================================================================================================================================================
\newpage
\section{Taxonomia}
\label{sec:taxonomia}
Com o objetivo de apresentar um panorama conceitual sobre o termo taxonomia e contextualizar o assunto, \citeonline{novo2007elaboraccao} diz:

\hspace{4cm}
\begin{minipage}{.66\textwidth}		
    \begin{singlespace}
        \fontsize{10}{12}\selectfont O conceito de taxonomia não emerge de repente como uma fórmula para solucionar problemas de representar o conhecimento de um dado domínio.
        É resultado de um longo processo histórico que se estendeu através de estudos e investigações que culminaram numa construção teórica.
        Esta evolução não foi linear tampouco ocorreu em um mesmo momento histórico, pois um único modelo não conseguiria responder de imediato a questões tão particulares 
        e complexas encontradas nas diversas áreas do conhecimento.
        \end{singlespace}
\end{minipage}
\vspace{0.3CM}

Taxonomia é, por definição, classificação, sistemática. Tradicionalmente utilizada para a classificação das espécies em botânica e zoologia, adotando uma definição binária.
\par
A utilização de taxonomia nos sistemas de informação não leva em consideração família, gênero ou espécie, mas sim conceitos.
As classes e subclasses de uma taxonomia se apresentam de maneira lógica, suportada por princípios classificatórios.\cite{campos2012taxonomia}
\par
Um dos recursos fundamentais para a gestão da informação e do conhecimento é o desenvolvimento de taxonomias \cite{dal2015ferramentas}.
\citeonline{aganette2010elementos} diz que diferente do princípio dicotômico adotado na taxonomia dos seres vivos, atualmente, faz-se possível a construção de taxonomias
policotômicas, ou seja, onde um objeto é classificado em tantas classes, e subclasses quantas necessárias, dentro um domínio especializado.
\par
Sendo assim, é possível utilizar a representação taxonômica da Figura \ref{fig:exemploTaxonomia} para classificar um objeto, arbitrariamente escolhido,
de acordo com os nós F e C, por exemplo.

\begin{figure}[!ht]
\begin{tikzpicture}[level 1/.style={sibling distance=5cm},level 2/.style={sibling distance=2.5cm}]
    \node {T}
        child { node {A}
        child {node{D}}
        child {node{E}}
        child {node{F}}
        child {node{G}}
        child {node{H}}
        }
        child { node {B}}   
        child { node {C}}  
    ;
\end{tikzpicture}
\caption{Exemplo de Taxonomia}
\label{fig:exemploTaxonomia}  
\end{figure}

%==================================================================================================================================================================================================
\subsection{Taxonomia para Ferramentas de Participação Eletrônica}
\label{subsec:taxonomia e-part tools}
\par
Por ser tratar de um assunto específico, inovador e relativamente novo, durante pesquisa para elaboração deste trabalho,
o único exemplo de taxonomia para ferramentas de participação eletrônica encontrado foi o esquema taxonômico elaborado pelos participantes do projeto de pesquisa Vispública, 
que será abordado na Seção \ref{subsec:taxonomiaElaborada}

%==================================================================================================================================================================================================
%\subsection{Taxonomia para Outras Ferramentas}
%\label{subsec:taxonomia other tools}

%==================================================================================================================================================================================================
\subsection{Taxonomia para Outras Áreas da Computação}
\label{subsec:taxonomia other computer areas}
Taxonomia, no contexto da computação, tem aplicação em distintas áreas. A utilização de estruturas taxonômicas para classificar sistemas, arquiteturas e arquivos data de mais de
cinquenta anos. Uma das classificações taxonômica com maior relevância para a área da computação é a chamada "Taxonomia de Flynn",
onde \citeonline{flynn1966very} classificou as arquiteturas de computadores da seguinte forma:\\

\begin{minipage}{.66\textwidth}
    \begin{singlespace}
        \begin{itemize}
            \item \acrfull{sisd}
            \item \acrfull{simd}
            \item \acrfull{misd}
            \item \acrfull{mimd}
        \end{itemize}
    \end{singlespace}
\end{minipage}
\vspace{0.5cm}

\begin{figure}[!ht]
    \begin{tikzpicture}[level 1/.style={sibling distance=5cm},level 2/.style={sibling distance=2.5cm}]
        \node {T}
            child { node {\acrshort{sisd}}}   
            child { node {\acrshort{simd}}}  
            child { node {\acrshort{misd}}}   
            child { node {\acrshort{mimd}}}  
        ;
    \end{tikzpicture}
    \caption{Taxonomia de Flynn}
    \label{fig:taxonomiaFlynn}  
\end{figure}

\vspace{0.5cm}
\par
A área de tolerância a faltas, da engenharia de \textit{software}, tem taxonomias comumente adotadas para definir termos e técnicas.
Entre as mais conhecidas estão a taxonomia proposta por \citeonline{gartner1999fundamentals}, abordando diversos conceitos e os aplicando a um cenário distribuído,
e a taxonomia apresentada por \citeonline{avizienis2004basic}, definindo conceitos sobre tolerância a faltas e segurança computacional.

\par
\citeonline{sondhi2018taxonomy} criou uma taxonomia que pode ser usada para prever as ações ou intenções de um usuário em particular de uma dada loja virtual
e então personalizar o algoritmo de busca para indicar as necessidades específicas desse usuário.

\par
Vale ressaltar a grande utilização de taxonomia por lojas virtuais, e o grande número de trabalhos encontrados sobre taxonomia aplicada a esse setor.

%==================================================================================================================================================================================================
\subsection{Taxonomia Elaborada}
\label{subsec:taxonomiaElaborada}
\par
Por meio do projeto Vispública foi elaborada a taxonomia mostrada na Figura \ref{fig:taxonomia-vispublica}. Neste trabalho, o esforço foi no sentido de fornecer a classificação de
ferramentas de participação através de uma taxonomia, de forma que seja possível associar uma ferramenta às dimensões propostas e também permitir uma compreensão
geral do que existe acerca das iniciativas de e-participação identificadas na literatura.

\begin{figure}[!ht]
    \centering
    \includegraphics[scale=0.3]{./figuras/taxonomia-cropped.png}
    \caption{Taxonomia elaborada pelo projeto Vispública}
    \label{fig:taxonomia-vispublica}
\end{figure}

