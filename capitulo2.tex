\chapter[Revisão Bibliográfica]{Revisão Bibliográfica} 
\label{cap:cap2}
% Introdução do capítulo
Neste capítulo são apresentados os tópicos teóricos necessários para uma boa compreensão do desenvolvimento deste trabalho.

% 1a seção do capítulo 1
\section{Participação Eletrônica}
\label{sec:e-part}
De acordo com a \acrfull{onu}, promover a participação cidadã é fundamental para a governança de uma sociedade inclusiva.
O objetivo dessa participação deve ser composto pela melhoraria do acesso à informação e serviços públicos,
e pelo incentivo a inclusão do cidadão nas tomadas de decisões públicas que impactem o bem estar da sociedade como um todo, e do indivíduo em particular. 
As funções de fazer política e entregar serviços públicos precisam ser reinterpretadas e os cidadãos têm que se envolver nos processos políticos \cite{bovaird2007beyond}. 
O uso de \acrfull{tic} no setor público tem transformado a governança global, nesse sentido, exigindo que governos forneçam
serviços melhores e mais eficientes a organizações e indivíduos \cite{afdb2014uneca}. O conceito de participação eletrônica foi definido por \citeonline{macintosh2008democracy} como:
"O uso de informações e comunicação tecnológicas a fim de ampliar e aprofundar a participação política para que cidadãos sejam capazes de se conectar 
uns aos outros e aos seus representantes eleitos".
A noção de governança digital é relatada por \citeonline{reddick2012public} como o resultado da adoção de infraestruturas tecnológicas para a otimização da entrega de serviços à população.
Ou seja, a utilização de \acrshort{tic} pelo setor público, passa a se referir ao modo como a internet pode melhorar a capacidade do Estado de formular
e implementar suas políticas \cite{parra2017governancca}. Sendo assim, \citeonline{germani2016desafios} complementa a definição de governança digital dizendo que:

\hspace{4cm}
\begin{minipage}{.66\textwidth}		
    \begin{singlespace}
        \fontsize{10}{12}\selectfont A utilização pelo setor público de recursos de tecnologia da informação e comunicação com o objetivo de melhorar a disponibilização
        de informação e a prestação de serviços públicos,
        incentivar a participação da sociedade no processo de tomada de decisão e aprimorar os níveis de responsabilidade, transparência e efetividade do governo.    
        \end{singlespace}
\end{minipage}

\vspace{0.3CM}
\par
Para \citeonline{braga2016participaccao} a possibilidade de um maior acesso as informações e ao conhecimento proporcionado pelas \acrshort{tic}
permite uma maior transparência nas decisões tomadas por governantes. A alegação que \citeonline{vaz2017transformaccoes} apresenta é a existência da necessidade de alteração do atual modelo
\textit{broadcasting} de governança eletrônica e quebra do monopólio estatal sobre as decisões e iniciativas de transparência e participação popular nas políticas públicas.
Para que haja a evolução desse modelo, \citeonline{o2011government} sugere que o governo deve disponibilizar suas informações em uma infraestrutura que permita sua sistemática reutilização
pela sociedade. Ao disponibilizar suas informação, o governo estimula o desenvolvimento de ferramentas tecnológicas e amplia a possibilidade 
do uso diverso das informações \cite{zuiderwijk2012socio}.

\section{Ferramentas de Participação Eletrônica}
\label{sec:e-part tools}
Segundo \citeonline{vaz2017transformaccoes}, o crescimento da comunidade \textit{open source} atrelado ao compartilhamento de dados governamentais abertos, permite a produção
descentralizada de aplicações, serviços e sistemas tecnológicos. Com isso, cria-se o que o mesmo autor chama de "segunda geração de governança eletrônica", que consiste na busca
por iniciativas além das unidirecionais, onde o governo apenas disponibiliza os dados captados sobre a população.

\par
O conceito de governança \textit{open source} democratiza a tomada de decisão e incentiva a colaboração voluntária entre os indivíduos \cite{rushkoff2003open}.
Diversos paradigmas, sobre o modelo de tomada de decisão, vêm sendo reexaminados, o papel do gestor público e do cidadão têm sido reavaliados,
e o uso das ferramentas computacionais se disseminado em muitos ambientes diferentes \cite{medeiros2009novos}.

\par
Devido aos incentivos gerados por esses novos paradigmas, muitas ferramentas vêm sendo desenvolvidas focadas nesse escopo.
Essas ferramentas podem ter aplicações direcionadas a cidades, bairros, regiões, estados e/ou países.  
Alguns exemplos podem ser encontrado na Tabela \ref{tab:ferramentas}.

\addtocounter{table}{-1}
\begin{table}[!ht]
    \centering
    \caption{Ferramentas de Participação Eletrônica}
    \label{tab:ferramentas}
    \begin{tabular}{l*{2}{>{\raggedright\arraybackslash}p{0.5\linewidth}}}
    \toprule
        Nome                             & Acesso                           \\ 
    \midrule
        Crowd For Roads                  & c4rs.eu                          \\
        Decidim Barcelona                & decidim.barcelona                \\
        e-Cidadania                      & senado.leg.br/ecidadania         \\
        Iniciativa de Cidadania Europeia & ec.europa.eu/citizens-initiative \\
        LabRIO                           & lab.rio                          \\
        Mandato Participativo            & saopaulo.sp.leg.br               \\
        Participa.BR                     & participa.br                     \\
        Patio                            & patiolla.fi                      \\
        Plataforma Brasil                & plataformabrasil.org.br          \\
        Portal e-Democracia              & edemocracia.camara.gov.br        \\
        SigaLei                          & sigalei.com.br                   \\
        Visor Urbano                     & visorurbano.com                  \\
        Vote na Web                      & votenaweb.com.br                 \\ 
        We The People                    & petitions.whitehouse.gov         \\
        WeLive                           & welive.eu                        \\
    \bottomrule
    \end{tabular}
\end{table}

As novas tecnologias mostraram uma gama de possibilidades para que os cidadãos ampliem o peso de sua participação nas decisões políticas,
melhorando a capacidade de mobilização, articulação, e possibilitando um maior envolvimento dos atores sociais \cite{araujo2015democracia}.\\

\par
\citeonline{saebo2008shape} dividem os atores sociais em quatro grupos: 

\begin{minipage}{.66\textwidth}	
   \textit{I}) Cidadãos, \\
   \textit{II}) Governantes, \\
   \textit{III}) Instituições estatais e \\
   \textit{IV}) Instituições Voluntárias. \\
\end{minipage}

\par
Esse agrupamento feito pelo autor supracitado é uma abstração da análise de \citeonline{macintosh2006evaluating}, deixando de lado um quinto grupo:

\par
\textit{IV}) Provedores de Tecnologia.

\par
Contudo, \citeonline{wimmer2007ontology} agrupa esses atores mais objetivamente, dividindo-os em apenas dois grupos, são eles:\\

\begin{minipage}{.75\textwidth}	
   \textit{a}) Beneficitários da utilização das ferramentas e \\
   \textit{b}) Responsáveis pela administração da ferramenta de participação.  \\
\end{minipage}

Esses grupos começaram a se utilizar de ferramentas de participação eletrônica, muitas vezes em projetos pilotos,
sem estratégias para medir os impactos gerados, as análises de performance ou as respostas dos cidadãos beneficiados \cite{macintosh2008democracy}.
Muitos autores chamam atenção para fato das ferramentas de participação serem predecessoras dos modelos de escolha e classificação para elas 
\cite{millard2006egovernance,macintosh2008democracy,reddick2012public}.

\par
A \acrfull{ocde}, juntamente com a \acrshort{onu} e a \acrfull{ue} criaram \textit{frameworks} para a avaliação das ferramentas através de indicadores. 
Atualmente, governantes podem encontrar diretrizes para seguir na hora de desenvolver uma atividade de participação eletrônica.

\section{Taxonomia}
\label{sec:taxonomia}

\subsection{Taxonomia para Ferramentas de Participação Eletrônica}
\label{subsec:taxonomia e-part tools}

\subsection{Taxonomia para Outras Ferramentas}
\label{subsec:taxonomia other tools}

\subsection{Taxonomia para Outras Áreas da Computação}
\label{subsec:taxonomia other computer areas}

\subsection{Taxonomia Elaborada}
\label{subsec:taxonmia made}